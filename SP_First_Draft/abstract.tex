%   Filename: abstract.tex 
\begin{abstract}
	
	
	\textit{Tegillarca granosa}, commonly known as blood cockles, is a significant marine bivalve species due to its nutritional benefits and economic importance. Accurate sex determination is essential for maintaining a balanced male-to-female ratio, sustainable harvesting, and resource management. However, the sex-determining mechanisms based on shell morphology are challenging macroscopically, and no existing technologies are available for non-invasive sex classification.  This study proposes the use of machine learning and deep learning techniques to classify the sex of blood cockles based on various shell measurements (length, width, height, hinge line distance, umbo distance, and rib count) and images from multiple camera angles (dorsal, ventral, anterior, posterior, and lateral views). The initial machine learning analysis using K-Nearest Neighbor (KNN) achieved an accuracy of 64.16\%, a precision of 64.97\%, a recall of 64.16\%, and an F1-score of 63.75\%. In contrast, deep learning with Convolutional Neural Networks (CNN) achieved an accuracy of 71.68\%, a precision of 72.52\%, a recall of 69.29\%, an F1-score of 69.12\%, and an AUC score of 77.34\% using images captured from the left lateral angle. These results offer a non-invasive method for sex identification, which could help in sustainable resource management and serve as a baseline for future studies on blood cockles classification.
	
	
	%  Do not put citations or quotes in the abstract.
	
	
	
	\begin{flushleft}
		\begin{tabular}{lp{4.25in}}
			\hspace{-0.5em}\textbf{Keywords:}\hspace{0.25em} & deep learning, supervised machine learning, computer vision, convolutional neural network, blood cockle, sex identification, \textit{Tegillarca granosa}\\
		\end{tabular}
	\end{flushleft}
\end{abstract}
