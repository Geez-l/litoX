%   Filename    : chapter_5.tex 
\chapter{Conclusion and Recommendations}


\section{Conclusion}

This study utilized the application of machine learning and deep learning techniques to identify the sex of \Tgranosa based on the morphometric characteristics. A manually curated dataset was developed, consisting of both linear measurements and images captured from six different angles. Machine learning methods were employed to identify statistically significant features, which served as the basis for deep learning analysis using a 12-layer Convolutional Neural Network (CNN). The proposed CNN model yielded an average accuracy of 71.68\% in the performance metrics. Overall, this study offers a classification approach which is a viable solution for non-invasive sex identification, providing an in-depth analysis based on \textit{T. granosa}’s linear measurements and morphological characteristics from different angles.

Through the availability of the gathered data, trial-and-error experimentation was conducted by adjusting the number of layers, batch size, epoch, and activation functions. The different combinations tested provided baseline results that demonstrate the feasibility of non-invasive sex identification for \textit{T. granosa}. 

While the study has made significant progress, challenges were encountered during CNN training, particularly due to hardware memory limitations. To overcome these, the researchers utilized synchronous Google Colab with 100 computing units, requiring subscriptions, repeated retraining, and reconfigurations, which demanded considerable financial resources and time to optimize the parameters. 

Upon comparing the experimental results of model parameters, it was demonstrated that non-invasive sex identification on \Tgranosa is achievable through the integration of machine learning and deep learning methods. Machine learning models based on five statistically selected features had better performances than those based on all features, with an accuracy of 64.16\%, precision of 64.97\%, recall of 64.16\%, and an F1-score of 63.57\% using K-nearest neighbors (KNN) classifier. The classification performance was further enhanced by deep learning models, using Left Lateral image view, achieving an accuracy of 71.68\%, precision of 72.52\%, recall of 69.29\%, F1-score of 69.12\%, and an AUC score of 77.34\%. 

These findings establish that the CNN model can serve as a baseline for future studies on non-invasive sex identification of \Tgranosa and potentially other similar species. By providing a practical and less harmful alternative to traditional methods, this research contributes a significant advancement in the field of aquaculture and marine biology.

\section{Recommendations}

This special problem entitled Morphometric and Morphological-Based Non-invasive Sex Identification of \Tgranosa focuses on creating a baseline study that will serve as a foundation for further studies involving \textit{T. granosa}, blood cockles, using machine learning, computer vision, and deep technologies in determining the sex of the samples is a salient need in aquaculture practices. Thus, the proposed recommendations are the future applications to improve and have detailed analysis, such as focusing on shape analysis, exploring other state-of-the-art deep learning techniques, or transfer learning, such as ResNet, SqueezeNet, and InceptionNet, and comparing the analysis results. Furthermore, the main goal of conducting this is to have the ability to identify the sex of the samples by taking real-time angles by rotating from the dorsal, lateral, and ventral.

Due to the time constraints, the researchers were only able to gather a total of 1,626 images with 271 images per angle, and utilized these for model training and validation. A larger and more diverse collection of images could further improve the model’s generalization. In order to capture more variability, future study might include expanding the dataset to improve classification performance. 

Future studies could also invest in a  sturdier and more controlled environment by using a green background and positioning a fixed camera angle during image acquisition. In addition, researchers may experiment with other image processing techniques such as morphological transformations to emphasize features. The dataset can be utilized for further analysis through advanced deep learning and computer vision methods to make sense of the images gathered and discern sexual dimorphism for \Tgranosa. 