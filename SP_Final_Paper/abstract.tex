\begin{center}
	\textbf{Abstract}
\end{center}
\setlength{\parindent}{0pt}
\textit{Tegillarca granosa}, commonly known as blood cockles, is a significant marine bivalve species due to its nutritional value and economic importance. Accurate sex identification is crucial for maintaining a balanced male-to-female ratio, supporting sustainable harvesting, and improving resource management. However, macroscopically identifying sex through shell morphology is challenging, and there are currently no available technologies for non-invasive sex classification. This study explores the use of machine learning and deep learning techniques to classify the sex of blood cockles based on shell measurements (length, width, height, hinge line length, distance between the umbos, and rib count) and images taken from various angles (dorsal, ventral, anterior, posterior, and lateral views). Machine learning analysis using K-Nearest Neighbors (KNN) achieved 64.16\% accuracy, 64.97\% precision, 64.16\% recall, and 63.75\% F1 Score. Moreover, deep learning using Convolutional Neural Networks (CNN) achieved 71.68\% accuracy, 72.52\% precision, 69.29\% recall, 69.12\% F1 Score, and 77.34\% AUC score using images captured from the left lateral angle view. These results demonstrate the potential of a non-invasive approach to sex identification, supporting sustainable aquaculture practices and offering a baseline for further research using computer vision and machine learning.

\begin{tabular}{lp{4.25in}}
	\hspace{-0.5em}\textbf{Keywords:}\hspace{0.25em} & deep learning, supervised machine learning, computer vision, convolutional neural network, blood cockle, sex identification, \textit{Tegillarca granosa}\\
\end{tabular}
