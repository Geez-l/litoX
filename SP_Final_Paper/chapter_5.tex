%   Filename    : chapter_5.tex 
\chapter{Conclusion and Recommendations}

\section{Conclusion}

This study aimed to develop a noninvasive approach for sex identification of \textit{Tegillarca granosa} using morphometric and morphological characteristics through the integration of machine learning and deep learning technologies. In particular, it sought to determine whether measurable shell features and image-based characteristics could reliably distinguish between male and female blood cockles.

The findings support the feasibility of this approach, with the proposed CNN model achieving a classification accuracy of 71.68\%. This performance demonstrates that linear morphological and features, when processed through deep learning, can serve as reliable indicators of sex in \textit{T. granosa}. In comparison to traditional, more invasive methods such as dissection or spawning observation, this method presents a promising alternative that could be useful in aquaculture operations requiring rapid and non-destructive sex identification.

The study also contributes a manually curated dataset of labeled images and shell measurements, which can serve as a foundation for further studies in this underexplored domain. By emphasizing noninvasiveness, the research addresses a crucial need in sustainable aquaculture practices, particularly in improving broodstock selection without harming specimens.

Although challenges such as limited sample size and computing resources were encountered, the overall results suggest that machine learning and deep learning techniques offer a scalable and practical solution for this biological classification task. As such, the study lays the groundwork for future research to expand the dataset, explore more advanced neural architectures, and develop real-time sex identification systems suitable for field or hatchery deployment.

\section{Recommendations}

This special problem aims to serve as a foundational study for future work involving the application of machine learning and deep learning in aquaculture. Given the importance of accurate sex identification for breeding and stock management, several recommendations are proposed to enhance future studies.

Future work should consider incorporating shape analysis and exploring more advanced deep learning architectures, such as ResNet, SqueezeNet, and InceptionNet. The use of transfer learning may also enhance classification performance, especially when working with limited datasets. Real-time sex identification could be achieved by developing a system that captures rotational views of the shell from dorsal, lateral, and anterior angles.

Due to time constraints, this study utilized a dataset of 1,626 images, with 271 images per angle. Increasing the number and diversity of samples can help improve model generalization and robustness. Expanding the dataset to include different populations and environmental contexts would provide a more comprehensive understanding of morphological variation in \textit{T. granosa}. Instead of manually gathering the linear measurements, another area of exploration would be on automated collection of measurements using images.

Moreover, future researchers are encouraged to establish a controlled image acquisition environment, using a green or neutral background, consistent lighting, and fixed camera positioning. Image processing techniques, including morphological transformations or background removal, should be further refined to highlight relevant features and enhance model accuracy.

The dataset produced in this study may serve as a valuable resource for future research in deep learning and marine biology. It can be further analyzed using advanced techniques to uncover patterns of sexual dimorphism and develop scalable, real-time applications for aquaculture settings.