%   Filename    : chapter_5.tex 
\chapter{Conclusion and Recommendations}


\section{Conclusion}


\section{Recommendations}

This special problem entitled Morphometric-Based Non-invasive Sex Identification of \textit{T.granosa} focuses on creating a baseline study that will serve as a foundation for further studies involving \textit{Tegillarca granosa}, blood cockles using machine learning, deep learning, and computer vision technologies in determining the sex of the samples which is a salient need in aquaculture practices. Thus, the proceeding recommendations are the future applications that can be further improved in analyze such as focusing on the shape analysis, exploring other state of the art CNN such as the ResNet, SqueezeNet, InceptionNet and compare the result of the analysis. Furthermore, the main goal of conducting this is to have the ability to real-time identify the sex of the samples by taking real-time angles by rotating from the dorsal, lateral, ventral. 

Future studies could also invest in a much sturdier and more controlled environment by using green background, using a web cam in a fix angle. It is also recommended to experiment with other image processing techniques such as scaling, rotating, and augmentation. 


