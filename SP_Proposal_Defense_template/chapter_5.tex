%   Filename    : chapter_5.tex 
\chapter{Conclusion and Recommendations}


\section{Conclusion}


\section{Recommendations}

This special problem entitled Morphometric-Based Non-invasive Sex Identification of \textit{T. granosa} focuses on creating a baseline study that will serve as a foundation for further studies involving \textit{Tegillarca granosa}, blood cockles using machine learning, deep learning, and computer vision technologies in determining the sex of the samples is a salient need in aquaculture practices. Thus, the proceeding recommendations are the future applications to improve and have detailed analysis such as focusing on shape analysis, exploring other state-of-the-art CNN such as ResNet, SqueezeNet, and InceptionNet, and comparing the analysis result. Furthermore, the main goal of conducting this is to have the ability to identify the sex of the samples by taking real-time angles by rotating from the dorsal, lateral, and ventral. 

Future studies could also invest in a much sturdier and more controlled environment by using a green background and positioning a webcam at a fixed angle. In addition, experiment with other image processing techniques such as scaling, rotating, and augmentation. The dataset can be utilized for further analysis using deep learning and computer vision to make sense of the images gathered and discern sexual dimorphism for \textit{T.granosa} or will serve as the basis for conducting similar studies to other bivalve species. 





