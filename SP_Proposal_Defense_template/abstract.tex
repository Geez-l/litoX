%   Filename    : abstract.tex 
\begin{abstract}


\textit{Tegillarca granosa} (Linnaeus, 1758), known as blood cockles is one of the most well-known marine bivalves due to its nutritional benefits and one of the major sources of livelihood. Due to these, it is crucial to determine their sex to maintain a balanced male-to-female ratio to avoid exploitation of this shellfish commodity. The sex-determining mechanism in the shell morphology of bivalves is challenging macroscopically due to the limited literature regarding this expertise. In addition,  no current technologies are employed to classify the sex based on shell morphology. This paper proposes a classification approach for the blood cockle's sex analyzed based on the linear measurements (length, width, height, distance hinge line, distance between umbos, rib count) and different angles (dorsal, ventral,  anterior, posterior, left and, right lateral). This study aims to discern the sexual dimorphism present between the male and female \textit{ T. granosa} utilizing supervised machine learning models for linear measurements (logistic regression, random forest, support vector machine, k-nearest neighbors, and extreme gradient boosting) as well as Convolutional Neural Networks (CNN) for image analysis using the widely used models for bivalve studies (VGGNet, Inception-ResNet, SqueezeNet). 















%  Do not put citations or quotes in the abract.



\begin{flushleft}
\begin{tabular}{lp{4.25in}}
\hspace{-0.5em}\textbf{Keywords:}\hspace{0.25em} & deep learning, supervised machine learning , convolutional neural network, blood cockle, sex identification, \textit{Tegillarca granosa}\\
\end{tabular}
\end{flushleft}
\end{abstract}
