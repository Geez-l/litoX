%   Filename    : chapter_1.tex 
\chapter{Introduction}
\label{sec:researchdesc}    %labels help you reference sections of your document

\section{Overview}
\label{sec:overview}

The Philippines is a global center of marine biodiversity and has established aquaculture as a significant contributor to total fishery production \cite{aypa2000, bfar2019}. The country produces over 4 million tonnes of seafood annually and is the 11th largest seafood producer in the world. Aquaculture is deeply integrated into Filipinos' livelihoods, encompassing fish cultivation and the production of various aquatic species, including bivalves. Among these, blood cockles \textit{(Tegillarca granosa)} hold considerable economic and environmental significance, essential to ensure sustainable production and population balance.

Maintaining a balanced male-to-female ratio of blood cockles is crucial to prevent overharvesting and ensure sustainable production. An imbalanced ratio can lead to overexploitation and impact the population's sustainability. However, there is limited literature on \textit{T. granosa} that has a thorough understanding of its sex-determining mechanisms, particularly concerning sexual dimorphism based on morphological and morphometric characteristics \cite{breton2017sex}.

Currently, sex determination methods for blood cockles are invasive, including dissection and histological examinations, which often result in the death of the species. While there is growing literature on aquaculture commodities sex identification using machine learning and deep learning, there is a notable scarcity of research addressing \textit{T. granosa} \cite{miranda2023}.

This study, titled "A Non-Invasive Sex Identification of \textit{T. granosa} using Machine Learning," aims detailed and baseline analysis of blood cockles by leveraging their morphological and morphometric characteristics. Sexual dimorphism in bivalves is hardly expressed and challenging to establish mascropically\cite{karapunar2021}. However, by integrating machine learning and deep learning, the study seeks to identify distinct features that may indicate sexual dimorphism between male and female blood cockles.

\section{Problem Statement}

Identifying the sex of \textit{T. granosa} is important to promote sustainable aquaculture and biodiversity by maintaining a balanced male-to-female ratio. A balanced ratio helps prevent overharvesting. Although sex identification is important for blood cockle population management and sustainable aquaculture, there is a notable lack of research in creating non-invasive methods to identify the sex of \textit{T. granosa}. Many of the latest studies and approaches are based on invasive methods like dissection or histological analysis, which are impractical for large-scale aquaculture operations focused on conservation.

The current methods for determining the sex of \textit{T. granosa} are invasive and  involve dissection, which involves cutting open the shell to visually inspect the gonads \cite{erica2018}. This procedure can cause harm to the specimens and frequently leads to their death. Another method is histological examination, where tissue samples are analyzed under a microscope \cite{may2021}. Both approaches are labor-intensive, time-consuming, and can pose risks to population management, particularly when maintaining a balanced sex ratio for breeding programs is essential. Moreover, stated invasive methods require specialized technical skills for accurate execution. Resource-limited settings in aquaculture operations encounter significant challenges in accessing the necessary laboratory equipment, such as microscopes and staining tools, which complicates the process.

A less invasive approach employed by aquaculturists is to monitor spawning behavior in which individuals are separated and stimulated to reproduce to determine their sex through the release of gametes \cite{miranda2023}. Although it is indeed less invasive than dissection, spawning still involves inducing stress in blood cockles and may not be completely effective for fast identification in large populations.

Given the limitations of both invasive and less invasive methods, this highlights the need for a more advanced approach. An alternative, non-invasive method involving machine and deep learning technologies might solve these issues by providing a fast, accurate, and effective solution without harming or stressing the blood cockles.

\section{Research Objectives}
\label{sec:researchobjectives}

\subsection{General Objective}
\label{sec:generalobjective}

The general objective of this study is to develop a non-invasive method for identifying the sex of \textit{Tegillarca granosa} using machine and deep learning integrated with computer vision technologies. This method aims to provide accurate and streamlined sex identification without causing harm to the specimens, thus supporting sustainable aquaculture practices.

\subsection{Specific Objectives}
\label{sec:specificobjectives}

To achieve the overall general objective of developing a non-invasive sex identification of \textit{T. granosa} using machine learning, deep learning, and computer vision technologies, the following specific objectives have been established:  

\begin{enumerate}
   \item To collect and organize a comprehensive dataset of \textit{T. granosa} which will include high-quality images and relevant morphological measurements that will serve as the basis for the machine-learning model.
 
   \item To develop and implement machine learning models that can classify the sex of \textit{T. granosa} based on the collected linear measurements and images of different angles of the sample.
   
   \item To evaluate the performance of the models used using performance metrics such as accuracy, precision, recall, and F1-score. 
   
   \item To develop a system that can identify the sex of \textit{T. granosa} based on its morphological characteristics. 
\end{enumerate}

\section{Scope and Limitations of the Research}
\label{sec:scopelimitations}

This study is conducted alongside the ongoing research by the UPV DOST-PCAARRD, titled "Establishment of the Center for Mollusc Research and Development: Development of Spawning and Hatchery Techniques for the Blood Cockle (\textit{Anadara granosa}) for Sustainable Aquaculture."The ongoing research primarily involves the rearing of \textit{T. granosa} from spat to larvae, as well as feeding experiments, stocking density evaluations, substrate selection, and settlement rate assessments.

In contrast, this study mainly focuses on developing a non-invasive method for identifying the sex of \textit{Tegillarca granosa} using machine learning, deep learning, and computer vision technologies. The goal is to provide an accurate and efficient means of sex identification without causing harm to the samples, contributing to sustainable aquaculture practices.

The researchers will work with 500 spawned blood cockles taken from Panay Island, specifically from Zarraga Iloilo and Ivisan Capiz, equally divided between 250 males and 250 females, obtained through induced spawning through temperature shock. The researchers will personally gather linear measurements, including length, width, height, rib count, length of the hinge line, and distance between the umbos using the vernier caliper. Images of the specimens, captured from various angles, will also be gathered under the supervision of University Research Associates from the Institute of Aquaculture, College of Fisheries and Ocean Sciences.

Data collection will take place at the hatchery facility of the University of the Philippines Visayas and will be conducted in batches, depending on the availability of spawned samples.

The method developed in this study is specific to \textit{Tegillarca granosa} and may not be applicable to other bivalve species. The model will be trained exclusively for \textit{Tegillarca granosa} and morphological features including length, width, height, rib count, length of the hinge line, and distance between the umbos may not be consistent across other shellfish species. 

\section{Significance of the Research}
\label{sec:significance}

This study will give us a significant advancement in non-invasive sex identification methods in \textit{T. granosa} providing innovative solutions that could solve the challenges in identifying sex and reshape sustainable approaches to aquaculture. The significance of this study extends to the following:

 \textit{Research Institution.} The result of this study focusing on the sex-identification mechanism of bivalves, specifically \textit{Tegillarca granosa}, will provide valuable insights into universities and research centers that focus on fisheries and coastal management, such as the UPV Institute of Aquaculture, that aim to develop sustainable development and suitable culture techniques.

 \textit{Fishermen.} By developing a non-invasive method in sex identification, this study can help long-term harvest efficiency and maintain the ratio of the harvest which can help prevent overexploitation of the \textit{T. granosa.}

 \textit{Coastal Communities.} The result of this study would be beneficial for the coastal communities that are reliant on their source of income with aquaculture commodities like blood cockles. Maintaining the diversity and aspect ratio of male and female may increase the market value of blood cockle production since cockle aquaculture faces significant obstacles worldwide due to the fluctuating seed supplies and scarcity of broodstock from the wild. 

 \textit{Future Researchers.} The result of this study would serve as the basis for studies that involve sex identification in bivalves such as \textit{T. granosa}. Some technologies are yet to be explored in machine learning, deep learning, and computer vision technologies that can lead to higher accuracy and distinguish the presence of sexual dimorphism in the \textit{T. granosa}.


