%   Filename: abstract.tex 
\begin{abstract}
	
	
	\textit{Tegillarca granosa} (Linnaeus, 1758), commonly known as blood cockles, is one of the most well-known marine bivalves for its nutritional benefits and economic significance. Determining their sex is essential for maintaining a balanced male-to-female ratio, which is crucial for preventing the exploitation of this shellfish resource. The sex-determining mechanism in the shell morphology of bivalves is challenging macroscopically due to the limited literature regarding this expertise. In addition, no current technologies are employed to classify the sex based on shell morphology. This study proposes a machine learning and deep learning approach for classifying the sex of blood cockles using various linear measurements (length, width, height, distance between the hinge line, distance between umbos, and rib count) and angles (dorsal, ventral, anterior, posterior, left lateral, and right lateral) collected from male and female specimens. Initial machine learning analysis aimed to determine the best-performing model and the significant features. Among the models, K-Nearest Neighbor (KNN) performed best, achieving an accuracy of 64.16\%, a precision of 64.97\%, a recall of 64.16\%, and an F1-score of 63.75\%. Feature importance analysis indicated that the Width-Height ratio was the most significant feature. Subsequently, deep learning analysis was conducted utilizing Convolutional Neural Networks (CNN), using the Left Lateral Angle, achieving an accuracy of 71.68\%, a precision of 72.52\%, a recall of 69.29\%, an F1-score of 69.12\%, an AUC score of 77.34\%. By developing a method to identify their sex, this study aims to improve the long-term availability of these marine resources and promote sustainable harvesting. 
	
	
	
	
	
	
	
	
	
	
	
	
	
	
	
	
	
	%  Do not put citations or quotes in the abstract.
	
	
	
	\begin{flushleft}
		\begin{tabular}{lp{4.25in}}
			\hspace{-0.5em}\textbf{Keywords:}\hspace{0.25em} & deep learning, supervised machine learning, computer vision, convolutional neural network, blood cockle, sex identification, \textit{Tegillarca granosa}\\
		\end{tabular}
	\end{flushleft}
\end{abstract}
