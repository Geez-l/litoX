%   Filename    : abstract.tex 
\begin{abstract}


\textit{Tegillarca granosa} (Linnaeus, 1758), commonly known as blood cockles, is one of the most well-known marine bivalve for its nutritional benefits and economic significance. Determining their sex is essential for maintaining a balanced male-to-female ratio, which is crucial for preventing overexploitation of this shellfish resource. The sex-determining mechanism in the shell morphology of bivalves is challenging macroscopically due to the limited literature regarding this expertise. In addition, no current technologies are employed to classify the sex based on shell morphology. This study proposes a machine learning approach for classifying the sex of blood cockles using various linear measurements (length, width, height, distance between the hinge line, distance between umbos, and rib count) and angles (dorsal, ventral, anterior, posterior, left lateral, and right lateral) collected from male and female specimens. Available machine learning models in MATLAB were trained to discern sexual dimorphism. Among the models, Linear SVM performed best, achieving an accuracy of 69.80\%, precision of 69.82\%, recall of 69.80\%, and an F1-score of 69.73\%. Feature importance analysis indicated that the distance between the umbos and height were the most significant features.















%  Do not put citations or quotes in the abract.



\begin{flushleft}
\begin{tabular}{lp{4.25in}}
\hspace{-0.5em}\textbf{Keywords:}\hspace{0.25em} & deep learning, supervised machine learning , convolutional neural network, blood cockle, sex identification, \textit{Tegillarca granosa}\\
\end{tabular}
\end{flushleft}
\end{abstract}
